\documentclass{resume}
\usepackage{multicol}
\rohead{last updated on \today}
\pagenumbering{gobble}
\begin{document}

%----------------------------------------------------------------------------------------
%	TITLE SECTION
%----------------------------------------------------------------------------------------

\name{Amal Krishna R}

\begin{center}
\small{
		\footerphone{+91 94970 59502} \
		\footerlink{http://amalrkrishna.github.io/} \
		\footeremail{amal.k.1994@ieee.org}
	}
\end{center}

\smallskip
\begin{multicols}{2}
%----------------------------------------------------------------------------------------
%	EDUCATION SECTION
%----------------------------------------------------------------------------------------

\section{Education}

\textbf{B.Tech in Avionics Engineering:} \hfill 2012 - 2016 \\
\href{http://iist.ac.in/}{Indian Institute of Space Science and Technology (IIST)},\\
Thiruvananthapuram,\\
India - [6.7/10 CGPA]\\[.05in]
\textbf{High School:} \hfill 2012\\ 
\href{http://www.stthomastvm.edu.in/central/index.aspx}{St Thomas Central School},\\ Thiruvanathapuram (CBSE) - [91.2\%]\\[.05in]
\textbf{Secondary School:} \hfill 2010\\  
St Thomas Central School,\\
Thiruvanathapuram (CBSE) - [9.2/10 CGPA]

%----------------------------------------------------------------------------------------
%	RELEVANT EXPERIENCE SECTION
%----------------------------------------------------------------------------------------

\section{Relevant Experience}

\textbf{Summer Intern,} \hfill May, 2015 - July, 2015\\
\textbf{Indian Institute of Space Science and Technology}\\
\emph{Mentored by \href{https://www.iist.ac.in/avionics/bsmanoj}{B.S. Manoj, Dept. of Avionics, IIST}}\\
Analyzed the challenges of Software Defined Network (SDN) in a high delay environment. An Software Defined Delay Tolerant Network (SDDTN) module is deployed onto every switch using OpenFLow protocol which gets activated when there is an absence of main controller connection. The module act as a light-weight controller which generates the flow for the switch and compute the plaussible locations to store the packets in the isolated network. \\

%----------------------------------------------------------------------------------------
%	TECHNICAL SKILLS SECTION
%----------------------------------------------------------------------------------------

\section{Technical Skills}

\textbf{Strongest Areas} - Software Defined Networks, Wireless Mesh Networks, Algorithms, Data Science.\\[.05in]
\textbf{Languages} - Python, C, C++, PHP, GNU Octave, Shell Script, JS, VHDL, Assembly Language\\[.05in]
\textbf{Tools/Frameworks} - MySQL, \LaTeX, Apache Spark, Git, OpenGL, RYU, Open vSwitch, OLSR daemon\\[.05in]
\textbf{Platforms} - Linux, Debian, Windows \\[.05in]
\textbf{IDE} - Eclipse, Netbeans, PHP Storm\\[.05in]
\textbf{Text Editor} - Geany, Gedit, Vim, Nano\\[.05in]

%----------------------------------------------------------------------------------------
%	PUBLICATIONS SECTION
%----------------------------------------------------------------------------------------

%\section{Publications}
%\begin{itemize}[leftmargin=*]
%\item Devanshu J, \textbf{Ashish K}, Rakshit S, Sameer S, ``Recommendation Techniques for %Adaptive E-learning'', Advances in Computer Science and Information Technology, vol. 2, No. %1, 2015. \href{https://drive.google.com/file/d/0B6A-3_6rwie9bS1OaFdzbW9BZXM/view?usp=sharing}%{view here}
%\item \textbf{Ashish Kedia} and Anusha Prakash, "Data Synchronization on Android Clients", %International Conference on Communication Software and Networks, June 6-7$^{th}$, 2015, %Chengdu, China. \href{http://ieeexplore.ieee.org/xpl/articleDetails.jsp?%reload=true&arnumber=7296156}{view here}
%\end{itemize}

%----------------------------------------------------------------------------------------
%	RELEVANT COURSE SECTION
%----------------------------------------------------------------------------------------

\section{Relevant Courses}

\textbf{Classroom} - Computer Networks, Wireless Mesh Networks, Data Structures and Algorithms, Virtual Reality, Computer Organization and Operating System, Information Theory and Coding.\\[.05in]
\textbf{MOOCs (certified)} - Algorithms: Design and Analysis, Part 1 (Stanford), Machine Learning (Stanford), Cryptography 1 (Stanford), Hadoop Platform and Application Framework (UC SanDiego), Python for Data Science (Microsoft).

%----------------------------------------------------------------------------------------
%	CONFERENCES/WORKSHOPS SECTION
%----------------------------------------------------------------------------------------

\section{Conferences/Workshops}

\href{http://raics.in/}{\textbf{IEEE International Conference on RAICS}} \hfill 2013\\
Thiruvanathapuram\\
\textbf{Raspberry Pi Workshop} \hfill 2014\\
Indian Institute of Space Science and Technology, Thiruvanathapuram\\
\textbf{An Insight Into THz Antenna Technology} \hfill 2016\\
Indian Institute of Space Science and Technology, Thiruvanathapuram\\

%----------------------------------------------------------------------------------------
%	INITIATIVES SECTION
%----------------------------------------------------------------------------------------

\section{Initiatives}

\textbf{IEEE Student Member}\\
\href{http://www.ieee.org/}{The Institute of Electrical and Electronics Engineers}\\[.05in]
\textbf{ACM Student Member}\\
\href{http://www.acm.org/}{Association for Computing Machinery}\\[.05in]
%\textbf{Student Volunteer} \hfill 2013\\
%Smile Foundation, NGO, Thiruvanathapuram
\textbf{Creativity Head} \hfill 2015\\
\href{https://www.facebook.com/conscientia.iist/}{\emph{Conscientia 2015}}, Annual Astronomical and Technical Festival, IIST\\[.05in]
\textbf{Finance and Creativity Head} \hfill 2014\\
\href{https://www.facebook.com/iist.dhanak/}{\emph{Dhanak 2014}}, Annual Cultural Festival, IIST\\[.05in]
\textbf{Publicity Co-Head} \hfill 2013\\
Dhanak 2013, Annual Cultural Festival, IIST\\[.05in]
\textbf{Web and Creativity Co-Head} \hfill 2013\\
Conscientia 2013, Annual Astronomical and Technical Festival, IIST\\

%----------------------------------------------------------------------------------------
%	Relevant Projects Section
%----------------------------------------------------------------------------------------
\section{Relevant Projects}
Projects available on git : \url{https://www.github.com/amalrkrishna}
%\setlist[itemize]{
\begin{itemize}
 \item \textbf{\href{}{On Switch-based Controller Hand-offs in Software Defined Wireless Mesh Networks}} : We use Expected Transmission Time (ETT) as the metric for controller hand-off in OpenFlow WMNs. ETT reflects various physical-layer characteristics, such as link traffic and end-to-end bandwidth. The experimental results showed that ETT is a better metric compared to RTT and ETX in a dynamic network with variable load across the links. ETT-based hand-off is able to respond to the excessive load in the link and make suitable hand-off decisons, whereas RTT and ETX fails in accomplishing the same with lower hand-off delay and packet dropouts.
 \item \textbf{\href{}{Software Defined MICRONet}} : A scaled down model of Software Defined MICRONet(Mobile Infrastructure for Costal Region Offshore Communications and Networks) environment was emulated. Software Defined MICRONet architecture provides intelligent communication among physical boat clusters in the sea. This will solve the technology challenges faced by the fishermen community in India today, specifically by providing software defined Intelligent and adaptable communication and connectivity while they are out at sea. 
 \item \textbf{\href{https://github.com/amalrkrishna/virtualnav-mpu6050}{
Navigation in a Virtual Environment using IMU MPU-6050
}} : Did a hardware implementation to navigate in a virtual environment developed in OpenGL using a low-cost Inertial Measurement Unit(IMU) MPU 6050.
 \item \textbf{\href{https://github.com/amalrkrishna/32bitrisc-vhdl}{32 bit RISC Microprocessor}} : The project was written in VHDL language and implemented on Altera FPGA. The Test bench module is executed in the Model-sim software and the LCD module is implemented on the FPGA to display the Register value, Memory value and the Program counter. 
\end{itemize}
\end{multicols}
\end{document}
